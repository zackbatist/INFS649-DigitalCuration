\documentclass{article}
\usepackage[utf8]{inputenc}
\usepackage{enumitem}
\usepackage[margin=1in]{geometry}


\title{Curating archaeological research data in practice, from the field to the archive}
% \subtitle{INF 649: Digital Curation}
\author{Zack Batist}
\date{November 4, 2024}

\begin{document}

\maketitle

\begin{itemize}
  \item Hi! Appreciation for being present so early.
  \item Quip about my experience in an iSchool Digital Curation class.
  \item Why I'm here: to speak about my research concerning the curation of research data
  \item Specifically: practical application of data work
  \item And some disjunctures between popular imagery and practical reality
  \item Caveat about being critical from a loving perspective, in relation to open science advocacy
\end{itemize}

\section{About me}
\subsection{Archaeologist}
\begin{itemize}
  \item My roots are in archaeology
  \item Excavation experience
  \item Experience applying digital and data-driven methods, e.g. network analysis
  \item Experience as database manager
  \item Experience managing open-archaeo
  \item Experience with archaeo-social
\end{itemize}

\subsection{Scholar of Scientific Practice}
\begin{itemize}
  \item These experiences inspired my work as a scholar of scientific practice
  \item Discrepancies between the vision and practical reality of the work
  \item And thinking about data management as people management, not just technical work
  \item So I began to think about open data as one form of data-sharing, which expands upon more local and restricted forms of collaboration that occur within projects
  \item And thinking about open data as posing its own challenges
  \item Specifically relating to how it re-shapes how knowledge production is distributed
  \item Or rather, how it \emph{attempts} to do this -- it's unclear whether things have actually changed that much
\end{itemize}

\section{Key questions and concerns}
\begin{itemize}
  \item So in a more general sense, my work examines how archaeologists participate in the development and use of information commons
  \item I'm especially interested in tracing the social and technical structures that scaffold data work and data-sharing, as these activities occur throughout the research process
\end{itemize}

\section{Key questions and concerns 2}

\begin{itemize}
  \item To do this work, I adopt a certain perspective put forward by my supervisor
  \item Specifically, that:
  \begin{itemize}
  \item All research activities involve interpretive decisions
  \item All activities take in and push out information, and effectively re-interpret prior meanings and re-present them in anticipation of future activities
  \item In this sense, data are connective tissue in a continuum of practice
  \item Or put another way, they are communicative media that enable translation of meanings across activity contexts
  \end{itemize}

\end{itemize}

\section{Notions of data-sharing}
\begin{itemize}
  \item And I consider that this is also true in the context of data-sharing,
  \begin{itemize}
    \item where the functional value of data is essentially extended beyond the scope of the projects from which they originate 
  \end{itemize}
  \item This perspective allows us to consider data as products of human decisions and actions
  \item And it also shines a light on the practical challenges involved in data work
  \item Numerous studies have shown that those who re-use data lack sufficient context about the circumstances of the datasets' creation
  \begin{itemize}
    \item And that they like to reach out through informal channels to understand how the data were created, beyond what is included in official documents
  \end{itemize}
  \item They also like knowing \emph{who} created the dataset, and judge the quality of the data based on the creator's reputation
  \item In this sense, data re-users want information that could only otherwise be understood through what effectively entails a close collaborative and social relationship
  \item Moreover, on the creators' end of the archive, it is common for researchers to do all their analysis in-house, before publishing a sucked orange with all its value already extracted
  \item This isn't just selfishness, but may be due to practical considerations
  \item They may have anticipated specific use-cases for their data, and their team may be best equipped to do the analyses themselves
  \item So, from all this, the relatively simplistic notion of uploading and downloading CSV files is actually more complicated in practice
  \item It's clear that social and professional norms influence \emph{how} data are re-used, and \emph{who} may re-use data
  \item And my research is concerned with highlighting these factors
\end{itemize}

\section{Methods}
\begin{itemize}
  \item I rely on a variety of data sources and analytical methods to do this
  \item I already mentioned my work that analyzes data pulled from coding and publishing platforms
  \item However, my dissertation work, which is what I'll be focusing on today, was much more in-depth
  \item It essentially involved observing and interviewing archaeologists while they worked, and examining the documents they produced
  \item I put cameras on archaeologists’ heads and in the corners of the trench or lab environment while they excavated or sorted through artefacts, and interviewed them about their work practices
  \item I also conducted numerous retrospective interviews to ascertain bigger picture values and priorities
  \item I was therefore able to compare what archaeologists said they were doing with how they were actually behaving, and trace connections between past and future activities recorded across different times and places
  \item I did this at three cases from 2016-2019;
  \item one case was longitudinal case over 3-4 years, and one was explicitly focused on a data archive
\end{itemize}

\section{Data Collection}
\begin{itemize}
  \item I wanted to look at how data are collected and transformed into more stable and transmissible media, so I'll talk about these aspects, in sequence.
\end{itemize}

\subsection{Recording Sheets}
\begin{itemize}
  \item The most acute and visible mode of information work within archaeological projects consists of acts of recording:
  \begin{itemize}
    \item filling in recording sheets and writing notes in a field journal
  \end{itemize}
  \item So one priority was to articulate recording practices and how these practices were situated within the broader apparatus of archaeological knowledge production
  \item Specifically, I found it valuable to compare the use of recording sheets and field journals,
  \begin{itemize}
    \item which afford different kinds of behaviours and communicative outcomes.
  \end{itemize}\vspace{1em}

  \item In my analysis of the context recorded sheet used at one of my cases, I identified five main sections.\\
  
  \item First:
  \begin{itemize}
    \item a section is dedicated to storing indexical information that identifies the locus, or excavation unit, that the sheet pertains to
    \item This includes unique identifiers for:
    \begin{itemize}
      \item the excavation unit
      \item trench
      \item survey unit
      \item or feature,
      \item as well as dates when these things were uncovered,
      \item and initials of the people who did the work
    \end{itemize}
  \end{itemize}\vspace{1em}

  \item Second:
  \begin{itemize}
    \item throughout the document,
    \item and clustered into subsections corresponding with different kinds of materials,
    \item the recording sheet prompts users to provide structured information according to a controlled vocabulary
  \end{itemize}
  \item Fields prompt users to record the things they found,
  \item the depth of the trench in various locations,
  \item the properties of the soil,
  \item and the equipment they used for excavation.\\
  
  \item The third section prompts users to describe the excavation unit in their own words.
  \item This allows them to highlight relationships among entities within the locus and among loci.\\
  
  \item The fourth section prompts users to relate the record with other media pertaining to the same archaeological entity,
  \item such as photographs, illustrations, orn related other documents.\\
  
  \item Finally, the fifth section contains a blank grid and relationship chart,
  \item where users could draw and identify the locus and surrounding loci in visual or schematic ways.\\
  
  \item The fieldworkers I spoke with perceived recording sheets as formal documents that are meant to contain official or authoritative accounts of each locus and its material properties.
  \item Consequently, new fieldworkers often felt a need to ask questions about how they should fill out these forms to meet the expectations of the project.
  \item In some ways, filling the recording sheets seemed to represent a somewhat bureaucratic obligation.\\
  
  \item While recording sheets were considered official records, they were sometimes also viewed as cumbersome obstacles that distract from ongoing work or that fail to capture what was really occurring in the trench.
  \item For instance, Theo, a seasoned fieldworker, was somewhat dismissive of the recording sheets and resentful of the demands that they impose.
  \item He believed that context sheets force him to write his observations in unnatural ways, forcing naturally fuzzy information into strict and arbitrary forms.
  \item For Theo, recording sheets are tools that warp reality into \emph{abstractions of reality}.
\end{itemize}

\subsection{Field Journals}
\begin{itemize}
  \item Archaeologists often compared recording sheets with field journals, which are the other primary way archaeologists record their observations in fieldwork settings.

  \item According to Theo, field journals record a ``stream of consciousness'' and provide a more genuine account of what occurred in the field.
  \item They enable a reader ``to understand what the excavator was thinking … whilst they were excavating''.\\
  
  \item In other words, they serve as mnemonic devices that preserve memories of the reasoning behind decisions that excavators made,
  \item but which they may forget during the flurry of activities that they must perform or that may fade from institutional or collective memory,
  \item as fieldworkers move on to other projects or otherwise become inaccessible.\\
  
  \item While Theo claimed that ``in the journal you can just write the fuck you want'' there are professional expectations that guide what information supervisors should record in field journals and how they should structure that information.
  \item As with recording sheets, the field journals I examined comprised a few distinct sections.\\
  
  \item First, they contain indexical information that identifies the general scope of the work, as well as information about who was responsible for leading or carrying out the work.
  \item This is typically on the cover page or the first page.
  \item Then, the journal entries follow.
  \item These are typically recorded on a day-by-day basis rather than ordered by unit or locus.
  \item Each entry may contain its own indexical information, such as the date, a list of people involved in the work, unique identifiers of contexts being worked on, etc.
  \item Entries also typically mention the conditions or circumstances under which work is occurring, such as the weather, remarks about the crew’s general attitude and morale, or any disruptions that may have occurred that day.
  \item They may also list the goals set out for each day of work, relating entries to each other and leading to the formation of quasi-narratives about work progress.
  \item The main content of journal entries consists of a log of decisions that the supervisor made and instructions to and carried out by assistants.
  \item They also include fleeting interpretations of phenomena being uncovered, revealing why and how certain decisions were made during the work process.
  \item Journal entries also commonly use colloquial language and refer to entities they recover in a very casual way.
  \item For instance, the journal entry depicted in this image refers to areas of the trench as the ``sand pit of doom'' and ``bouldery hell''.\\
  
  \item The field journals I examined are crafty, multi-media documents.
  \item They often contain sketches or schematic visualizations of the trench, of the landscape, of relevant features, or of mental models scattered throughout the notebook.
  \item Sketches are without scale, and entities are labelled only when the illustrator deems it necessary at the time of drawing.
  \item They also sometimes contain hand-drawn tables recording regularly formatted data, such as running lists of photographs taken, contexts opened, special finds and their spatial coordinates, or samples taken.
  \item Because these tables are typically recorded at the end of the notebook and are filled in as new pertinent info comes across their radar, they tend either to run out of space or to reserve too many extra pages.
  \item Sometimes, tabs are added to the edges of those pages using a piece of paper reinforced with scotch tape to make them easier to access. Notebooks sometimes have pages ripped out or have pages informally added with tape, glue, or a stapler.\\
  
  \item The journal entries switched between atomic and descriptive characterizations of specific elements within the trench and more speculative associations that draw the trench within a broader understanding of the site as a whole.
  \item They exhibit greater flexibility than more formal records in that they often refer to a variety of related entities or observations on the basis of the judgement and experience of the writer.
  \item In this way, field journals are discursive media that describe and discuss particular aspects of the project from the situated perspectives of their authors and contextualize and define an object’s significance on the basis of particular experiences with it.\\
  
  \item Trench supervisors sometimes elaborated on these rough interpretations during site tours, which, at least at one of my cases, were regularly scheduled events whereby the whole team went around the site to learn what was going on in each trench.
  \item When the team arrived at a trench, its supervisor described its principal features, typically in a fashion that recalls the work and decisions involved in its exploration.
  \item Usually, the project director or analysts supplemented this account by making interjections or rebuttals, helping to situate the trench in relation to broader project-wide narratives.
  \item Site tours were informal and were never recorded, but they conveyed a great deal of information to listeners.
  \item Tours used imprecise language and referred to things whose meanings may not have been well understood outside the project team.
  \item For instance, members of Case A often referred to the “red shit,” which signifies a layer of red clay that appears throughout the site and which nearly all excavators have had to struggle with.\\
  
  \item Project directors also liked to give these tours to visiting scholars, notable guests, and new project participants so that they could get a better understanding of what was going on in the site, rather than being limited to what was published in a paper or report.
\end{itemize}

\subsection{Comparison}
\begin{itemize}
  \item This echoes other mentioned statements made by my informants regarding the value of personal and informal modes of communication when trying to relate the character of a site to those who are less familiar with it.

  \item In each case, there was a general consensus among fieldworkers that journals captured much more information than recording sheets, though of a different kind.

  \item This is especially interesting in light of the fact that the information contained in journals occupy are rarely transcribed as more formal records,
  \item and even occupy an entirely separate data stream as that which gets processed into a project's relational database.
\end{itemize}

\section{Digital Transformation}
\begin{itemize}
  \item Which brings me to thinking about how these records and transformed into digital media
  
  \item As archaeological data are collected, it is necessary to render them in ways that are more amenable to systematic analysis

  \item And of course, this is typically achieved by inputting and organizing data using digital systems such as relational databases, file systems, and digital archives.
\end{itemize}

\subsection{Databases}
\begin{itemize}
  \item Databases served to:
  \begin{itemize}
    \item centralize data,
    \item relate the outputs generated by complementary streams of investigation,
    \item and ensure that the data are structurally consistent.
  \end{itemize}\vspace{1em}

  \item The databases used by the projects I examined were custom-built and used conventions specific to the project.
  \item Practical decisions about the database were often made ``on the fly'' or were derived through trial and error.
  \item Database managers often learned their skills on the job and assembled code that was previously published on various blogs, tutorials, and online forums.\\
  
  \item At the same time, database managers often struggled to reconcile the information presented by these disparate sources, and the products they eventually cobbled together did not always perform optimally.\\
  
  \item So the databases are in one sense representations of a project's priorities, and in another sense manifest the meandering journey of the individual who put it all together.
\end{itemize}

\subsection{Messy Records}
\begin{itemize}
  \item I also observed that formal data contained within databases,
  \begin{itemize}
    \item which are characterized by being clean and tidy,
    \item and are arranged so that they are more conducive to complex retrieval queries and patterned analysis,
    \item often originated as relatively messy analog records that are more amenable to fieldwork conditions.
  \end{itemize}\vspace{1em}

  \item Through data entry and data-cleaning processes, the values written down on paper recording sheets were copied to homologous and homogenous digital tables.\\
  
  \item However, fieldwork documentation was performed in ways that were responsive to that specific work environment,
  \item and did not actively account for those transformations that would occur down the line.\\
  
  \item For instance:
  \begin{itemize}
    \item fieldworkers used imperfect spelling and grammar,
    \item used shorthand representations,
    \item deviated from controlled vocabularies,
    \item and crossed out and re-wrote text.
  \end{itemize}\vspace{1em}

  \item According the database managers I spoke to, they were responsible for correcting somewhat trivial errors,\\
  \item like different spelling of words referring to the same thing.\\
  
  \item This aspect of their job involved transcribing written records into formats optimized for computer-assisted data retrieval.\\
  
  \item This sometimes involved significant editing and omission of information contained on the handwritten records.
  \item Words or values that have been crossed out or revised were not copied over;
  \begin{itemize}
    \item different handwriting or penmanship, which implies different authors or circumstances under which the records were made, were disregarded;
    \item and drafted versions of recording sheets, which were entirely re-written, sometimes never made their way to the database manager at all.
  \end{itemize}\vspace{1em}
  
  \item Additionally, some elements that are difficult to represent as distinct database records,
  \begin{itemize}
    \item like sketches or mind-maps,
    \item were excluded from the database altogether.
  \end{itemize}\vspace{1em}
  
  \item Acts of transcription therefore involved a significant amount of transformation, including information loss.
\end{itemize}

\subsection{Anxiety}
\begin{itemize}
  \item And this often produced a sense of anxiety,
  \item since this work failed to meet the initial expectation of a smooth and frictionless workflow that database managers seemed to expect going in.\\

  \item For instance:
  \begin{itemize}
    \item Jamie -- who was one of the database managers for the primary case I've been discussing today --
    \item recognized that fieldwork should be modified to support analysis by producing cleaner and tidier records.
    \item She specifically advocated for implementing and enforcing standards on fieldwork activities to ensure that the data were more amenable to analytical purposes down the line.
  \end{itemize}\vspace{1em}

  \item Similarly:
  \begin{itemize}
    \item Paul, who maintained an archaeological data archive,
    \item considered it his job to help projects conceive if their research as data --
    \item by which he means as concrete and formally-modelled records.
  \end{itemize}\vspace{1em}

  \item These statements imply a perceived disconnect between the rough and improvised experience of fieldwork and a conception of what constitutes ``proper'' research and "proper" data.
\end{itemize}

\subsection{Digital Archives}
\begin{itemize}
  \item Projects sometimes hire archaeological data services to help maintain their data,
  \item with an eye toward curating, preserving, and publishing the data after the project is complete.\\

  \item By paying digital archives to curate their data,
  \item archaeological projects effectively delegate responsibility to sanitize, document, preserve, and distribute their data to dedicated experts who are committed to these tasks.\\

  \item Project leaders stated that depositing data in a digital archive also satisfied projects’ commitments to funding agencies,
  \item who often mandate that funded projects plan for proper and long-term care of their research materials,
  \item which includes ensuring that all data are publicly accessible.\\

  \item Interestingly, while digital archives’ role in data reuse is often touted as their primary function and benefit,
  \item the project directors I spoke with generally considered this a secondary concern.\\

  \item Altogether, my informants stated that digital curation services enable them to move forward with new projects without having to worry about the state of their prior work.

  \item So from one perspective, archives served as a kind of final resting place,
  \item and from another perspective they were places where old data could be given a new lease on life.\\

  \item But as I said earlier, this notion of archives as loci for simply uploading and downloading spreadsheets is a bit of a myth,
  \item and there will always be a need for discursive engagement to enable practical re-use.\\

  \item People who re-use data know, on an intuitive level, that there is more to the data than what is documented in the supplementary materials,
  \item and they make efforts to circumvent these systems to get at the information that will actually support their research.\\

  \item It's really ironic, then, that the formal and transactional protocols meant to streamline mutual comprehension of a dataset reveal their own inadequacy for achieving their stated purpose, while also revealing the strengths of the system that they are meant to replace, e.g. socially-mediated forms of collaboration.

  \item But archivists' focus on the technical processes of cleaning and documenting data shields them from grappling with this tension.
\end{itemize}

\section{Take-Aways}
\begin{itemize}
  \item So, overall, the database, and the supporting tools, media and actions that surround the database, are effectively strategies for ensuring that records are more amenable for analysis and for reducing traces of subjectivity
  \item However, subjective and informal communication persist as fundamental aspects of archaeological research, including in contexts of data-sharing and re-use\\
  
  \item Formal representations are lossy simplifications of much richer engagements with phenomena and objects of interest;
  \item And when sharing data, which effectively mediate between different activity systems, we need to ensure that mutual understanding is maintained
  \item This is the value proposition of digital curation, right?\\
  
  \item And yet, this community-driven aspect of data-sharing is still overlooked
  \item And it probably accounts for the low rates of re-use in archaeology, and in other disciplines too\\
  
  \item I hate to end on such a bleak note, so perhaps it might be best to consider this an area in which we need to work to improve
  
\end{itemize}

\end{document}